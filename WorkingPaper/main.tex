\documentclass[11pt]{article}
\usepackage{epsfig}
\usepackage[english]{babel}
\usepackage[utf8]{inputenc}
\usepackage{siunitx}
\usepackage{amsmath}
\usepackage{graphicx, fullpage}
\usepackage[colorinlistoftodos]{todonotes}
\usepackage{xspace}
\usepackage{hyperref}
\usepackage{lineno}
\usepackage{float}
\newcommand{\numu}{$\nu_\mu$\xspace}
\newcommand{\plus}{\ensuremath{\texttt{+}}}
\newcommand{\CCpionreaction}{$\ensuremath{\nu_{\mu}}~\plus~N~\rightarrow \mu^{\mp} \plus~N~\plus~\pi^\pm + X ~$}
\newcommand{\uscore}{\ensuremath{\texttt{$_$}}}
\newcommand{\CCpion}{\ensuremath{\pi^\pm}}
\newcommand{\numubar}{$\bar{\nu}_\mu$\xspace}
\newcommand{\nue}{$\nu_e$\xspace}
\newcommand{\nuebar}{$\bar{\nu}_e$\xspace}
\usepackage{multirow}
\usepackage{color}
\definecolor{my}{RGB}{220,230,140} 
\title{Sebastian's \numu CC 0$\pi$ Analysis}% Force line breaks with \\

\author{Sebastian Sanchez-Falero}

\date{\today}% It is always \today, today,
             %  but any date may be explicitly specified

\begin{document}
%\setpagewiselinenumbers
%\modulolinenumbers[1]
\linenumbers
\maketitle

%%%%%%%%%%%%%%%%%%%%%%%%%%%%%%%%%%%%%%%%%%%%%%%%%%%%%%%%%%%%%%%%%%%%%%%%%%%%%%
\begin{abstract}
X is interesting.  We have measured X.  The results are Y.
\end{abstract}


%%%%%%%%%%%%%%%%%%%%%%%%%%%%%%%%%%%%%%%%%%%%%%%%%%%%%%%%%%%%%%%%%%%%%%%%%%%%%%
\section{Introduction}





%%%%%%%%%%%%%%%%%%%%%%%%%%%%%%%%%%%%%%%%%%%%%%%%%%%%%%%%%%%%%%%%%%%%%%%%%%%%%%
\section{Datasets}






%%%%%%%%%%%%%%%%%%%%%%%%%%%%%%%%%%%%%%%%%%%%%%%%%%%%%%%%%%%%%%%%%%%%%%%%%%%%%%
\section{Preliminary Strategy}
We would start by making the following cuts:

\begin{enumerate}
   	\item Require a muon: Has a prong with muonID higher than \textit{value}
       \item Exclude multiple muons/pion-like tracks: reject events with other prongs longer than \textit{value}
       \item Exclude showers: reject events with high EM Id score
\end{enumerate}

%%ReCo NuE for events passing cuts, broken in signal/BKG 
%%Fractions os S/B for signal in a table, for different cut values. Could use TMVA

%%To get confidence, vary cuts, if overall shape varies the same in data/MC that gives confidence in the MC

%%Overall goal: Do signal optimization minimizing not S/(S+B) but systematic uncertainty.

%%Usually, statistics are assessed by tuning MC in a multiverse 

%%NOvA systematics: the xsection is pne the biggest systematics is the oscillatio analysis. Things like calibration

%%Prong Multiplicity for true signal /  BKG events Do ur cut affect the prong multiplicity? Are we getting rid of events that could be signal

%%Start wirth MC. Plot the total reco nu energy. then look for which events have true pions
%%Apply all the cuts and figure out which remaining events have pions
%%Total calo energy
%%PLot prong multiplicity for those cuts 
\noindent Of the remaining sample, we will classify by Number of Additional Prongs and make Data/MC comparisons of:
\begin{itemize}
	\item Prong multiplicity
    \item Prong Energies
    \item ???
\end{itemize}

\noindent In addition we will study how the variations in the cut values affect the purity and efficiency of selecting non-muon prongs


At each step, we need to understand what the excluded/passing events really constitute $\Rightarrow$ study efficiencies of each cut, in muon and other related particle variables
\begin{enumerate}
	\item Muon Prong and ID efficiencies and purities: Should be very well studied for the oscillation analyses
    \item Pion Prong and ID efficiencies and purities: Have began studies on this direction
\end{enumerate}


%%%%%%%%%%%%%%%%%%%%%%%%%%%%%%%%%%%%%%%%%%%%%%%%%%%%%%%%%%%%%%%%%%%%%%%%%%%%%%
\section{Event Selection}


%%%%%%%%%%%%%%%%%%%%%%%%%%%%%%%%%%%%%%%%%%%%%%%%%%%%%%%%%%%%%%%%%%%%%%%%%%%%%%
\section{Systematic Uncertainties}

%%%%%%%%%%%%%%%%%%%%%%%%%%%%%%%%%%%%%%%%%%%%%%%%%%%%%%%%%%%%%%%%%%%%%%%%%%%%%%
\section{Resolution and Binning}


%%%%%%%%%%%%%%%%%%%%%%%%%%%%%%%%%%%%%%%%%%%%%%%%%%%%%%%%%%%%%%%%%%%%%%%%%%%%%%
\section{Results, Comparisons and World Domination}



%%%%%%%%%%%%%%%%%%%%%%%%%%%%%%%%%%%%%%%%%%%%%%%%%%%%%%%%%%%%%%%%%%%%%%%%%%%%%%
\newpage
\section{Scratch*}
\begin{itemize}
	
    
    \item Signal definition: 
    	\begin{itemize}
    		\item Experimentally observable, as opposed to theoretical cross-section (QE) $\rightarrow$ less model-dependent
           	\item Can do comparisons to pure QE hypothesis to study nuclear effects
            \item Muon and not-pion are required. Defined above  thresholds:
    			\begin{itemize}
    				\item Reco eff (making a prong/track)
                    \item PID eff
    			\end{itemize}
                Study these efficiencies with montecarlo, in different muon and pion variables. Set thresholds in the plateau region, safely away from rising edge
            
    	\end{itemize}
    
    
    \item Reco and PID 
    	\begin{itemize}
    		\item p,$\mu$, $\pi$ can look all the same, for suitable energies
    		\item Useful tools:
    			\begin{itemize}
    				\item ProngCVN
                    \item dE/dx
                    \item Michel electrons
                    \item Annihilation photons
                    \item Displaced vertices
                    \item Dimuons
    			\end{itemize}
             \item Can develop own tools for RHC, use neutron studies
        \end{itemize}
	
    \item Preselection
    
    
    \item Selection
    	\begin{enumerate}
    		\item Find a muon: prong with muon ID $>$ sth
            \item Find a second prong. 
    	\end{enumerate}
    
    
    \item Possible Deliverables
    	\begin{itemize}
    		\item Single differential cross section in muon/proton variables 
    		\item Double differential cross section in muon/proton variables
            \item Hammer events
            \item FHC/RHC ratio? (long term)
	\end{itemize}
    
\end{itemize}



%%%%%%%%%%%%%%%%%%%%%%%%%%%%%%%%%%%%%%%%%%%%%%%%%%%%%%%%%%%%%%%%%%%%%%%%%%%%%%
\section{TODO}
\begin{itemize}
	\item Document reading
    \begin{itemize}
   		\item NOvA Charged Pion \cite{ccpiArisDPFtalk} \cite{ccpiArisDPFblessingpack}, \cite{ccpiArisNDTalk1}, \cite{ccpiArisNDTalk2}
        \item NOvA Proton ID \cite{protonIDMerritt}
        \item MINERvA Multinucleon effects \cite{mnvWCRodriguesMultinucleon}, \cite{Rodrigues:2015hik}, 
        \item MINERvA Antineutrino cross sections \cite{Patrick:2018gvi} 
    \end{itemize}
    

\end{itemize}





%%%%%%%%%%%%%%%%%%%%%%%%%%%%%%%%%%%%%%%%%%%%%%%%%%%%%%%%%%%%%%%%%%%%%%%%%%%%%%
% Bibliography
%\bibliographystyle{unsrt}
\newpage
\bibliographystyle{unsrtnat}
\bibliography{bibl}

\end{document}
%
% ****** End of file apssamp.tex ******